% Created 2017-02-20 Mon 18:00
% Intended LaTeX compiler: pdflatex
\documentclass[presentation]{beamer}
\usepackage[utf8]{inputenc}
\usepackage[T1]{fontenc}
\usepackage{graphicx}
\usepackage{grffile}
\usepackage{longtable}
\usepackage{wrapfig}
\usepackage{rotating}
\usepackage[normalem]{ulem}
\usepackage{amsmath}
\usepackage{textcomp}
\usepackage{amssymb}
\usepackage{capt-of}
\usepackage{hyperref}
\usetheme{default}
\author{Petr Blaho}
\date{\today}
\title{PA200 - Cloud Providers}
\hypersetup{
 pdfauthor={Petr Blaho},
 pdftitle={PA200 - Cloud Providers},
 pdfkeywords={},
 pdfsubject={},
 pdfcreator={Emacs 25.1.1 (Org mode 9.0.5)}, 
 pdflang={English}}
\begin{document}

\maketitle

\begin{frame}[label={sec:org657d1a9}]{Recap of Cloud Types}
\begin{itemize}
\item More detailed in Cloud Service Deliver Models
\item Datacenter Virtualization -- oVirt
\item IaaS -- Infrastructure as a Service -- OpenStack
\item PaaS -- Platform as a Servise - OpenShift
\item SaaS -- Software as a Service -- Microsoft Dynamics CRM -- mostly
abandoned
\item xPaaS -- extended PaaS -- integrates aPaaS (Application), iPaaS
(Integration), dvPaaS (Data Virtualization), bpmPaaS (Business
Process Management), mPaaS (Mobile)
\end{itemize}
\end{frame}

\begin{frame}[label={sec:org75e9755}]{Amazon}
\begin{block}{AWS}
\begin{itemize}
\item Amazon Web Services
\item Web service backend providing libraries for majority of modern web programming languages
\item Provides also database access and storage
\item Features mobile platform as well
\item REST-style HTTP, SOAP
\end{itemize}
\end{block}

\begin{block}{S3}
\begin{itemize}
\item Simple Storage Service
\item Provides API-driven Object storage
\item Stored files are abstracted all the way to objects and are easy to
represent in high-level programming languages
\item REST-style HTTP, SOAP, BitTorrent
\end{itemize}
\end{block}

\begin{block}{EC2}
\begin{itemize}
\item Elastic Compute Cloud (IaaS)
\item KVM-based
\item Provides backend to earlier mentioned services as well as ability to
sell „Virtual private cloud“ to customers
\item OSs (Linux, OpenSolaris, Windows, NetBSD, \ldots{})
\end{itemize}
\end{block}

\begin{block}{Other}
\begin{itemize}
\item Elastic Beanstalk (PaaS)
\item AWS Lambda (event-driven, serverless computing - FaaS)
\item Glacier (storage for archives or backups)
\item Elastic Block Store (block-level storage)
\item SimpleDB, DynamoDB, Elastic MapReduce, \ldots{}
\end{itemize}
\end{block}
\end{frame}

\begin{frame}[label={sec:org370a00e}]{Google}
\end{frame}

\begin{frame}[label={sec:org32b7a10}]{Cloud Platform}
\begin{itemize}
\item Rich platform exposing functionality over REST
\item Primarily developed to support Google's core services (search, youtube, gmail)
\item Later extended with the business needs driven by Android, its integration with services
\item Nowadays featuring rich set of programming frameworks, hosting services and database engines
\end{itemize}
\end{frame}

\begin{frame}[label={sec:org2ca57da}]{App Engine}
\begin{itemize}
\item Web apps on Google's infrastructure (PaaS)
\item Python, Java (JVM), Go, PHP, Node.js
\item Easy deploy, monitoring, scaling
\item Limited languages and tools (SQL vs. GQL)
\end{itemize}
\end{frame}

\begin{frame}[label={sec:org5cd3470}]{Compute Engine}
\begin{itemize}
\item IaaS
\item Compute Engine Unit (GCEU) - abstraction of computation power
\item at the backend kvm based
\end{itemize}
\end{frame}

\begin{frame}[label={sec:orgbbdfc86}]{Storage}
\begin{itemize}
\item IaaS for storage
\item REST-like HTTP access
\item compatible with Amazon S3
\end{itemize}
\end{frame}

\begin{frame}[label={sec:org66c057f}]{BigQuery}
\begin{itemize}
\item web service (with REST-like interface)
\item work with Storage
\item SQL dialect, returns JSON
\item can be integrated via HTTP (Spreadsheets)
\end{itemize}
\end{frame}

\begin{frame}[label={sec:org88e2426}]{Red Hat}
\begin{itemize}
\item Provider of solutions that can serve either as a private or public
cloud
\item Also provider of PaaS/xPaaS solution (OpenShift)
\item Involved in development of cloud-oriented apps ranging from Level 1
(kernel, KVM), through management software (OpenStack, oVirt) and
PaaS up to application level (Jboss Enterprise Application Platform,
Data Virtualization, etc.)
\end{itemize}
\end{frame}

\begin{frame}[label={sec:orgc7719c6}]{oVirt}
\begin{itemize}
\item open source upstream for Red Hat Virtualization
\item can manage networks, CPUs, storages
\item with VM it can do live migration, live snapshots
\item integrate with many open source projects (OpenStack, Foreman, ManageIQ, \ldots{})
\end{itemize}

\begin{block}{oVirt Engine}
\begin{itemize}
\item Java (GWT, WildFly)
\item REST-style HTTP API
\item can integrate with LDAP or AD
\end{itemize}
\end{block}

\begin{block}{oVirt Node}
\begin{itemize}
\item RHEL, CentOS, Fedora or Debian with KVM
\item VDSM (Python daemon) manages resources and VMs
\item gets commands from Engine and reports back to it
\end{itemize}
\end{block}
\end{frame}

\begin{frame}[label={sec:org9d8f1fe}]{OpenStack}
\begin{itemize}
\item open source platform for cloud computing (mainly IaaS)
\item written in Python
\item each Project aims to solve one part of cloud computing needs
\item pluggable w/r/t backends and between Projects
\end{itemize}

\begin{block}{Identity (Keystone)}
\begin{itemize}
\item central user management and authentication service
\item can use directory service backend (LDAP)
\end{itemize}
\end{block}

\begin{block}{Compute (Nova)}
\begin{itemize}
\item layer on top of hypervisor(s)
\item manages compute resources - VMs and containers
\end{itemize}
\end{block}

\begin{block}{Networking (Neutron)}
\begin{itemize}
\item manages networks and IP addresses for VMs
\item can use SDN technologies (OpenFlow)
\item load balancing, floating IPs, firewall, VPN, \ldots{}
\end{itemize}
\end{block}

\begin{block}{Other}
\begin{itemize}
\item Block Storage (Cinder) - many storage providers
\item Image Storage (Glance) - images to boot from and to store snapshots of VMs to
\item Orchestration (Heat) - used to manage deployments of applications on OS
\item Database as a Service (Trove)
\item Bare Metal (Ironic) - management of physical machines (PXE and IPMI as default)
\end{itemize}
\end{block}
\end{frame}

\begin{frame}[label={sec:orgb69f6a1}]{OpenShift}
\begin{itemize}
\item PaaS
\item container based deployment and management
\item Kubernetes with Docker images
\item written in Go
\end{itemize}
\end{frame}

\begin{frame}[label={sec:org6cf8733}]{Recap}

\end{frame}
\end{document}
